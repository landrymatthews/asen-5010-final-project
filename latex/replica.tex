\documentclass[12pt]{article}
\usepackage[margin=1in]{geometry}
\usepackage{amsmath,amsfonts,amssymb}
\usepackage{graphicx}
\usepackage{titlesec}
\usepackage{fancyhdr}
\usepackage{hyperref}
\usepackage{enumitem}
\usepackage{lscape}
\usepackage{booktabs}

\titleformat{\section}{\large\bfseries}{\thesection}{1em}{}
\titleformat{\subsection}{\normalsize\bfseries}{\thesubsection}{1em}{}

\pagestyle{fancy}
\fancyhf{}
\rhead{ASEN 5010}
\lhead{Spacecraft Dynamics Capstone Project}
\rfoot{\thepage}

\begin{document}

% Title Page
\begin{titlepage}
    \centering
    \vspace*{2cm}
    {\Huge \textbf{ASEN 5010 Semester Project}}\\[1.5cm]
    {\Large \textbf{Attitude Dynamics and Control of a}}\\[0.5cm]
    {\Large \textbf{Nano-Satellite Orbiting Mars}}\\[2cm]
    {\normalsize Prof. Dr. Vishala Arya}\\
    {\normalsize Aerospace Engineering Sciences}\\
    {\normalsize University of Colorado, Boulder}\\[2cm]
    {\large Mid-Project (Tasks 1--6) Report Due Date: April 10th 2024}\\
    {\large Final Project Report Due Date: April 24th 2024}
    \vfill
\end{titlepage}

\tableofcontents
\newpage

\section{Project Overview}
This project considers a small satellite orbiting Mars at a low altitude gathering science data. However,
this small satellite needs to transfer this data to a larger mother satellite at a higher altitude. Further, to
keep the batteries from draining all the way, periodically the satellite must point its solar panels at the
sun to recharge. Thus, three mission goals must be considered by the satellite: 1) point sensor platform
straight down at Mars, 2) point communication platform at the mother satellite, and 3) point the solar
arrays at the sun. In all scenarios the small spacecraft and mother craft are on simple circular orbits
whose motion is completely known.

\section{Mission Description}
A nano-satellite is on a circular low Mars orbit (LMO) to observe the non-sunlit Mars surface. A second satellite, the mother spacecraft, is on a circular geosynchronous Mars orbit (GMO). The nano-satellite must either:
\begin{itemize}
  \item point a sensor at the surface when in science mode,
  \item point the solar panels in the sun direction when in power mode, or
  \item point the communication dish at the mothercraft when in communication mode.
\end{itemize}

\section{Mission Scenario Definitions}
\subsection{Spacecraft Attitude States}
\begin{align}
\sigma_{B/N}(t_0) &= \begin{bmatrix} 0.3 \\ -0.4 \\ 0.5 \end{bmatrix} \\
{}^B\omega_{B/N}(t_0) &= \begin{bmatrix} 1.00 \\ 1.75 \\ -2.20 \end{bmatrix}~\text{deg/s} \\
{}^B[I] &= \begin{bmatrix} 10 & 0 & 0 \\ 0 & 5 & 0 \\ 0 & 0 & 7.5 \end{bmatrix}~\text{kg m}^2
\end{align}

\subsection{Mission Pointing Scenarios}
During the mission, the control law must detumble the satellite and point:
\begin{itemize}
  \item antenna (\(-\hat{b}_1\)) towards GMO,
  \item sensor (\(\hat{b}_1\)) at Mars (nadir or \(-\hat{r}\)),
  \item solar panels (\(\hat{b}_3\)) at the sun (\(\hat{n}_2\)).
\end{itemize}

\begin{table}[h!]
\centering
\caption{Spacecraft Pointing Scenario Summary}
\begin{tabular}{@{}ll@{}}
\toprule
Orbital Situation & Primary Pointing Scenario Goals \\
\midrule
SC on sunlit Mars side & Point Solar Panels axis $\hat{b}_3$ at Sun \\
SC not on sunlit Mars side \& GMO Visible & Point Antenna axis $-\hat{b}_1$ at GMO \\
SC not on sunlit Mars side \& GMO not Visible & Point sensor axis $\hat{b}_1$ along Mars nadir direction \\
\bottomrule
\end{tabular}
\end{table}

\subsection{Mission Orbit Overview}
\begin{table}[h!]
\centering
\caption{Initial Orbit Frame Orientation Angles}
\begin{tabular}{@{}cccc@{}}
\toprule
Spacecraft & $\Omega$ & $i$ & $\theta(t_0)$ \\
\midrule
LMO & $20^\circ$ & $30^\circ$ & $60^\circ$ \\
GMO & $0^\circ$ & $0^\circ$ & $250^\circ$ \\
\bottomrule
\end{tabular}
\end{table}

\subsection{Attitude Control Law Overview}
The PD control law is:
\begin{equation}
{}^B\!u = -K\sigma_{B/R} - P\,{}^B\omega_{B/R}
\end{equation}
where $K$ and $P$ are scalar gains.

\subsection{Astrodynamics Numerical Integration}
Use a 4th order Runge-Kutta (RK4) routine. Hold the control piecewise constant during the RK4 integration from $t_n$ to $t_{n+1}$. The pseudo-code is as follows:

\begin{verbatim}
X0 = [ σ_{B/N}(t0) Bω_{B/N}(t0) ];
tmax = ...; ∆t = ...; tn = 0.0;
while tn < tmax do
  if new control required then
    Evaluate [R_N(t)], Nω_{R/N}(t);
    Compute σ_{B/R}, Bω_{B/R};
    Determine control u;
  end
  k1 = ∆t*f(Xn, tn, u);
  k2 = ∆t*f(Xn + k1/2, tn + ∆t/2, u);
  k3 = ∆t*f(Xn + k2/2, tn + ∆t/2, u);
  k4 = ∆t*f(Xn + k3, tn + ∆t, u);
  Xn+1 = Xn + 1/6 * (k1 + 2k2 + 2k3 + k4);
  if |σ_{B/N}| > 1 then
    map σ_{B/N} to shadow set;
  end
  tn+1 = tn + ∆t;
  save X and u;
end
\end{verbatim}

\noindent Continue with task descriptions and equations as needed...
\subsection{Task 1: Orbit Simulation (5 points)}
Assume the general orbit frame \(\mathcal{N}\) is Mars-centered inertial. Propagate the orbit of the LMO nano-satellite using the provided orbit elements. Propagate for at least one full orbit. Plot a 3D view of the orbit trajectory. Then propagate the orbit of the GMO spacecraft and overlay this trajectory with the LMO orbit. Annotate a few position vectors of interest.

\subsection{Task 2: Orbit Frame Orientation (5 points)}
Using the orbit propagation of Task 1, compute the orbit frame \(\mathcal{H} = \{\hat{r}, \hat{\theta}, \hat{h}\}\). You can compute this from the numerical position and velocity vectors at each time step. Use the following DCM definition:
\[
[NH] = \begin{bmatrix} \hat{r} & \hat{\theta} & \hat{h} \end{bmatrix}
\]
Plot the unit vector components of \(\hat{r}, \hat{\theta}, \hat{h}\) as a function of time.

\subsection{Task 3: Sun-Pointing Reference Frame Orientation (10 points)}
Compute the DCM \([NR]\) that points the \(\hat{b}_3\) body axis toward the Sun. This DCM is computed using a sun vector that is known in inertial frame. The DCM \([NR]\) can be constructed using:
\begin{itemize}
  \item \(\hat{r}_3 = \) Sun vector normalized
  \item \(\hat{r}_2 = \hat{r}_3 \times \hat{h}\) (\(\hat{h}\) from orbit frame)
  \item \(\hat{r}_1 = \hat{r}_2 \times \hat{r}_3\)
\end{itemize}
Normalize the vectors and construct the full DCM. Then compute the MRP attitude error \(\sigma_{B/R}\) using a provided \(\sigma_{B/N}\) and \([NR]\).

\subsection{Task 4: Nadir-Pointing Reference Frame Orientation (10 points)}
Construct \([NR]\) for a frame that points the \(\hat{b}_1\) body axis toward the nadir direction \( -\hat{r} \). Use \(\hat{r}\) from Task 2. Construct a full orthonormal frame \(\{\hat{r}_1, \hat{r}_2, \hat{r}_3\}\) with \(\hat{r}_1 = -\hat{r}\) and the other axes constructed to lie in the orbital plane. Normalize all vectors and construct the DCM \([NR]\). Compute and plot \(\sigma_{B/R}(t)\) for one orbit.

\subsection{Task 5: GMO-Pointing Reference Frame Orientation (10 points)}
Compute the position vector from the LMO satellite to the GMO satellite at each time step. Define \(\hat{r}_1\) to point along this direction. Construct \(\hat{r}_2\) and \(\hat{r}_3\) using cross-products with known orbital elements. Normalize the resulting vectors to form a proper DCM \([NR]\). Plot the MRP attitude error \(\sigma_{B/R}(t)\) for the full orbit.

\subsection{Task 6: Attitude Error Evaluation (10 points)}
Use \(\sigma_{B/N}\) and \(\omega_{B/N}\) from the Project Overview. Assume \([NR]\) and \(\omega_{R/N}\) are known for one of the earlier defined scenarios (Sun, Nadir, GMO). Compute \(\sigma_{B/R}\) and \(\omega_{B/R} = \omega_{B/N} - [BN]\omega_{R/N}\). Then compute the control torque using the PD law:
\[
{}^B\!u = -K\sigma_{B/R} - P\,{}^B\omega_{B/R}
\]
Use \(K = 5\) and \(P = 30\). Show your work and all intermediate results.

\subsection{Task 7: Numerical Attitude Simulator (10 points)}
Write a simulator to propagate the spacecraft's rotational state using RK4. The state includes \(\sigma_{B/N}\) and \(\omega_{B/N}\). The dynamics are:
\begin{align}
\dot{\sigma}_{B/N} &= \frac{1}{4} \left( (1 - \sigma^T\sigma)I + 2[\sigma]_\times + 2\sigma\sigma^T \right) \omega_{B/N} \\
I\dot{\omega}_{B/N} &= -\omega_{B/N} \times I\omega_{B/N} + u
\end{align}
Simulate the motion over one orbit, using a constant control torque (e.g., from Task 6) or zero torque for debugging. Plot \(\sigma_{B/N}(t)\) and \(\omega_{B/N}(t)\).

\subsection{Task 8: Sun Pointing Control (10 points)}
Use the numerical simulator to compute control torque \(u(t)\) based on a time-varying \([NR](t)\) from Task 3. Simulate the PD controller applied during sun-pointing. Plot the control torque, MRPs, and body rate over time. Make sure to verify that the controller achieves sun-pointing and suppresses oscillations.

\subsection{Task 9: Nadir Pointing Control (10 points)}
Modify the controller to track the nadir-pointing frame from Task 4. Apply this during a different portion of the orbit and analyze transition behavior. Verify the controller tracks the nadir frame without significant overshoot or divergence. Plot \(\sigma_{B/R}(t)\), \(u(t)\), and \(\omega_{B/N}(t)\).

\subsection{Task 10: GMO Pointing Control (10 points)}
Implement the controller for the GMO-pointing scenario from Task 5. Run a segment of simulation with this reference and demonstrate proper convergence. Ensure the spacecraft points the \(-\hat{b}_1\) axis at the GMO. Plot relevant data to support your results.

\subsection{Task 11: Mission Scenario Simulation (10 points)}
Combine all control laws and reference frames into one simulation. Implement logic to switch between control modes based on time and orbital position. Simulate a full 6500-second scenario. Plot:
\begin{itemize}
  \item \(\sigma_{B/N}(t)\)
  \item \(\omega_{B/N}(t)\)
  \item Control torque \(u(t)\)
\end{itemize}
Annotate the plot to show when mode transitions occur (Sun, Nadir, GMO). Ensure the controller transitions correctly and all objectives are achieved.

\end{document}

\end{document}


\documentclass[conf]{new-aiaa}
\usepackage{graphicx}
\usepackage{amsmath}
\usepackage[version=4]{mhchem}
\usepackage{siunitx}
\usepackage{longtable,tabularx}
\setlength\LTleft{0pt}

\title{Attitude Dynamics and Control of a Nano-Satellite Orbiting Mars}

\author{Landry R. Matthews\footnote{Graduate Student, Aerospace Engineering Sciences, University of Colorado Boulder.}}
\affil{University of Colorado Boulder, Boulder, CO, 80309}

\begin{document}

\maketitle

\begin{abstract}
This document outlines the work completed for Tasks 1 through 6 of the ASEN 5010 Capstone Project. These tasks focus on simulating and analyzing the orbit and attitude dynamics of a nano-satellite in Mars orbit. Results are derived using a Python-based simulation framework and validated through analytical and numerical approaches.
\end{abstract}

\section{Introduction}
The project involves a nano-satellite in a circular Low Mars Orbit (LMO), performing various attitude maneuvers depending on whether it is in sunlit, communication, or science mode. The satellite must align its body-fixed reference frames accordingly using different pointing strategies such as sun-pointing, nadir-pointing, or GMO (Geostationary Mars Orbit) mothership communication alignment. Each of the following tasks contributes to a component of this overall attitude control simulation.

\section{Task 1: Orbit Simulation}
Using circular orbital dynamics, the inertial position and velocity vectors of the satellite are computed. The Python implementation utilizes a function \texttt{orbit\_sim()} that takes orbital elements and uses a 3-1-3 Euler rotation sequence to convert to inertial coordinates. The directional cosine matrix (DCM) is calculated via the function \texttt{Euler313toDCM()}.

\subsection*{LMO at $t = 450$ s}
\[
\vec{r}_{LMO} = 
\begin{bmatrix}
-669.29 \\ 3227.50 \\ 1883.18
\end{bmatrix} \quad \text{(km)}, \quad
\vec{v}_{LMO} = 
\begin{bmatrix}
-3.256 \\ -0.798 \\ 0.210
\end{bmatrix} \quad \text{(km/s)}
\]

\subsection*{GMO at $t = 1150$ s}
\[
\vec{r}_{GMO} = 
\begin{bmatrix}
-5399.15 \\ -19697.64 \\ 0.0
\end{bmatrix} \quad \text{(km)}, \quad
\vec{v}_{GMO} = 
\begin{bmatrix}
1.397 \\ -0.383 \\ 0.0
\end{bmatrix} \quad \text{(km/s)}
\]

\section{Task 2: Orbit Frame Orientation}
This task calculates the orientation of the orbit frame \textit{O} with respect to the inertial frame \textit{N}. The transformation is done through a 3-1-3 Euler angle sequence using symbolic algebra in SymPy to represent $\theta(t)$, the true anomaly as a function of time. The DCM $[HN]$ is computed and evaluated at $t = 300$ s.

\[
HN(t = 300s) = 
\begin{bmatrix}
-0.0465 & 0.8741 & 0.4834 \\
-0.9842 & -0.1229 & 0.1277 \\
0.1710 & -0.4698 & 0.8660
\end{bmatrix}
\]

\section{Task 3: Sun-Pointing Reference Frame Orientation}
When the spacecraft is on the sunlit side (positive $n_2$ coordinate), it must align its +b\textsubscript{3} axis (solar panel normal) with the sun, assumed to be in the +n\textsubscript{2} direction. To build an orthonormal frame, the -n\textsubscript{1} axis is used to define +b\textsubscript{1}.

\[
R_{sN}(t = 0s) = 
\begin{bmatrix}
-1 & 0 & 0 \\
0 & 0 & 1 \\
0 & 1 & 0
\end{bmatrix}, \quad
\vec{\omega}_{Rs/N} = 
\begin{bmatrix}
0 \\ 0 \\ 0
\end{bmatrix}
\]

\section{Task 4: Nadir-Pointing Reference Frame Orientation}
In science mode on the shadowed side of Mars, the satellite must point its sensor (+b\textsubscript{1}) directly toward the nadir direction (toward Mars center). The reference frame is constructed from the relative position vector and the local orbit track direction.

\[
R_{nN}(t = 330s) = 
\begin{bmatrix}
0.0726 & -0.8706 & -0.4866 \\
-0.9826 & -0.1461 & 0.1148 \\
-0.1710 & 0.4698 & -0.8660
\end{bmatrix}, \quad
\vec{\omega}_{Rn/N} = 
\begin{bmatrix}
0.000151 \\ -0.000416 \\ 0.000766
\end{bmatrix}
\]

\section{Task 5: GMO-Pointing Reference Frame Orientation}
In communication mode, the satellite aligns its -b\textsubscript{1} (antenna direction) with the direction of the GMO mothership. The relative vector between the spacecraft and GMO is used to compute the DCM $R_{cN}$, and angular velocity $\omega_{RC/N}$ is evaluated both analytically and numerically for verification.

\[
R_{cN} =
\begin{bmatrix}
0.2655 & 0.9609 & 0.0784 \\
-0.9639 & 0.2663 & 0.0 \\
-0.0209 & -0.0755 & 0.9969
\end{bmatrix}
\]

\[
\omega_{RC/N} (\text{Numerical}) =
\begin{bmatrix}
1.978 \times 10^{-5} \\
-5.465 \times 10^{-6} \\
1.913 \times 10^{-4}
\end{bmatrix}, \quad
\omega_{RC/N} (\text{Analytical}) =
\begin{bmatrix}
1.978 \times 10^{-5} \\
-5.465 \times 10^{-6} \\
1.913 \times 10^{-4}
\end{bmatrix}
\]

\section{Task 6: Attitude Error Evaluation}
Attitude error is evaluated using Modified Rodrigues Parameters (MRPs) and relative angular velocity between the body frame and reference frames. These are computed using functions like \texttt{DCM2MRP} and vector subtraction logic in \texttt{values.py}.

\subsection*{Sun-Pointing}
\[
\sigma_{B/R} =
\begin{bmatrix}
-0.7754 \\ -0.4739 \\ 0.0431
\end{bmatrix}, \quad
\omega_{B/R} =
\begin{bmatrix}
0.01745 \\ 0.03054 \\ -0.03840
\end{bmatrix}
\]

\subsection*{Nadir-Pointing}
\[
\sigma_{B/R} =
\begin{bmatrix}
0.2623 \\ 0.5547 \\ 0.0394
\end{bmatrix}, \quad
\omega_{B/R} =
\begin{bmatrix}
0.01685 \\ 0.03093 \\ -0.03892
\end{bmatrix}
\]

\subsection*{GMO-Pointing}
\[
\sigma_{B/R} =
\begin{bmatrix}
0.0170 \\ -0.3828 \\ 0.2076
\end{bmatrix}, \quad
\omega_{B/R} =
\begin{bmatrix}
0.01730 \\ 0.03066 \\ -0.03844
\end{bmatrix}
\]

\section{Task 7: Numerical Attitude Simulator}
\vspace{1em}
\textit{[Placeholder for Task 7 results.]}

\section{Task 8: Sun Pointing Control}
\vspace{1em}
\textit{[Placeholder for Task 8 results.]}

\section{Task 9: Nadir Pointing Control}
\vspace{1em}
\textit{[Placeholder for Task 9 results.]}

\section{Task 10: GMO Pointing Control}
\vspace{1em}
\textit{[Placeholder for Task 10 results.]}

\section{Task 11: Mission Scenario Simulation}
\vspace{1em}
\textit{[Placeholder for Task 11 results.]}

\section{Conclusion}
This report presents a comprehensive simulation of a nano-satellite’s orbit and attitude dynamics around Mars. Tasks 1 through 6 have been implemented and validated using analytical and numerical tools. Future work involves integrating control solutions and testing full mission scenarios.

\end{document}
